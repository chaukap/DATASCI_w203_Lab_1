% Options for packages loaded elsewhere
\PassOptionsToPackage{unicode}{hyperref}
\PassOptionsToPackage{hyphens}{url}
%
\documentclass[
]{article}
\usepackage{lmodern}
\usepackage{amssymb,amsmath}
\usepackage{ifxetex,ifluatex}
\ifnum 0\ifxetex 1\fi\ifluatex 1\fi=0 % if pdftex
  \usepackage[T1]{fontenc}
  \usepackage[utf8]{inputenc}
  \usepackage{textcomp} % provide euro and other symbols
\else % if luatex or xetex
  \usepackage{unicode-math}
  \defaultfontfeatures{Scale=MatchLowercase}
  \defaultfontfeatures[\rmfamily]{Ligatures=TeX,Scale=1}
\fi
% Use upquote if available, for straight quotes in verbatim environments
\IfFileExists{upquote.sty}{\usepackage{upquote}}{}
\IfFileExists{microtype.sty}{% use microtype if available
  \usepackage[]{microtype}
  \UseMicrotypeSet[protrusion]{basicmath} % disable protrusion for tt fonts
}{}
\makeatletter
\@ifundefined{KOMAClassName}{% if non-KOMA class
  \IfFileExists{parskip.sty}{%
    \usepackage{parskip}
  }{% else
    \setlength{\parindent}{0pt}
    \setlength{\parskip}{6pt plus 2pt minus 1pt}}
}{% if KOMA class
  \KOMAoptions{parskip=half}}
\makeatother
\usepackage{xcolor}
\IfFileExists{xurl.sty}{\usepackage{xurl}}{} % add URL line breaks if available
\IfFileExists{bookmark.sty}{\usepackage{bookmark}}{\usepackage{hyperref}}
\hypersetup{
  pdftitle={Lab 1: Question 3},
  pdfauthor={Chandler Haukap, Hassan Saad and Courtney Smith},
  hidelinks,
  pdfcreator={LaTeX via pandoc}}
\urlstyle{same} % disable monospaced font for URLs
\usepackage[margin=1in]{geometry}
\usepackage{color}
\usepackage{fancyvrb}
\newcommand{\VerbBar}{|}
\newcommand{\VERB}{\Verb[commandchars=\\\{\}]}
\DefineVerbatimEnvironment{Highlighting}{Verbatim}{commandchars=\\\{\}}
% Add ',fontsize=\small' for more characters per line
\usepackage{framed}
\definecolor{shadecolor}{RGB}{248,248,248}
\newenvironment{Shaded}{\begin{snugshade}}{\end{snugshade}}
\newcommand{\AlertTok}[1]{\textcolor[rgb]{0.94,0.16,0.16}{#1}}
\newcommand{\AnnotationTok}[1]{\textcolor[rgb]{0.56,0.35,0.01}{\textbf{\textit{#1}}}}
\newcommand{\AttributeTok}[1]{\textcolor[rgb]{0.77,0.63,0.00}{#1}}
\newcommand{\BaseNTok}[1]{\textcolor[rgb]{0.00,0.00,0.81}{#1}}
\newcommand{\BuiltInTok}[1]{#1}
\newcommand{\CharTok}[1]{\textcolor[rgb]{0.31,0.60,0.02}{#1}}
\newcommand{\CommentTok}[1]{\textcolor[rgb]{0.56,0.35,0.01}{\textit{#1}}}
\newcommand{\CommentVarTok}[1]{\textcolor[rgb]{0.56,0.35,0.01}{\textbf{\textit{#1}}}}
\newcommand{\ConstantTok}[1]{\textcolor[rgb]{0.00,0.00,0.00}{#1}}
\newcommand{\ControlFlowTok}[1]{\textcolor[rgb]{0.13,0.29,0.53}{\textbf{#1}}}
\newcommand{\DataTypeTok}[1]{\textcolor[rgb]{0.13,0.29,0.53}{#1}}
\newcommand{\DecValTok}[1]{\textcolor[rgb]{0.00,0.00,0.81}{#1}}
\newcommand{\DocumentationTok}[1]{\textcolor[rgb]{0.56,0.35,0.01}{\textbf{\textit{#1}}}}
\newcommand{\ErrorTok}[1]{\textcolor[rgb]{0.64,0.00,0.00}{\textbf{#1}}}
\newcommand{\ExtensionTok}[1]{#1}
\newcommand{\FloatTok}[1]{\textcolor[rgb]{0.00,0.00,0.81}{#1}}
\newcommand{\FunctionTok}[1]{\textcolor[rgb]{0.00,0.00,0.00}{#1}}
\newcommand{\ImportTok}[1]{#1}
\newcommand{\InformationTok}[1]{\textcolor[rgb]{0.56,0.35,0.01}{\textbf{\textit{#1}}}}
\newcommand{\KeywordTok}[1]{\textcolor[rgb]{0.13,0.29,0.53}{\textbf{#1}}}
\newcommand{\NormalTok}[1]{#1}
\newcommand{\OperatorTok}[1]{\textcolor[rgb]{0.81,0.36,0.00}{\textbf{#1}}}
\newcommand{\OtherTok}[1]{\textcolor[rgb]{0.56,0.35,0.01}{#1}}
\newcommand{\PreprocessorTok}[1]{\textcolor[rgb]{0.56,0.35,0.01}{\textit{#1}}}
\newcommand{\RegionMarkerTok}[1]{#1}
\newcommand{\SpecialCharTok}[1]{\textcolor[rgb]{0.00,0.00,0.00}{#1}}
\newcommand{\SpecialStringTok}[1]{\textcolor[rgb]{0.31,0.60,0.02}{#1}}
\newcommand{\StringTok}[1]{\textcolor[rgb]{0.31,0.60,0.02}{#1}}
\newcommand{\VariableTok}[1]{\textcolor[rgb]{0.00,0.00,0.00}{#1}}
\newcommand{\VerbatimStringTok}[1]{\textcolor[rgb]{0.31,0.60,0.02}{#1}}
\newcommand{\WarningTok}[1]{\textcolor[rgb]{0.56,0.35,0.01}{\textbf{\textit{#1}}}}
\usepackage{graphicx,grffile}
\makeatletter
\def\maxwidth{\ifdim\Gin@nat@width>\linewidth\linewidth\else\Gin@nat@width\fi}
\def\maxheight{\ifdim\Gin@nat@height>\textheight\textheight\else\Gin@nat@height\fi}
\makeatother
% Scale images if necessary, so that they will not overflow the page
% margins by default, and it is still possible to overwrite the defaults
% using explicit options in \includegraphics[width, height, ...]{}
\setkeys{Gin}{width=\maxwidth,height=\maxheight,keepaspectratio}
% Set default figure placement to htbp
\makeatletter
\def\fps@figure{htbp}
\makeatother
\setlength{\emergencystretch}{3em} % prevent overfull lines
\providecommand{\tightlist}{%
  \setlength{\itemsep}{0pt}\setlength{\parskip}{0pt}}
\setcounter{secnumdepth}{-\maxdimen} % remove section numbering

\title{Lab 1: Question 3}
\author{Chandler Haukap, Hassan Saad and Courtney Smith}
\date{}

\begin{document}
\maketitle

\hypertarget{are-people-who-believe-that-science-is-important-for-making-government-decisions-about-covid-19-more-likely-to-disapprove-of-the-way-their-governor-is-handling-the-pandemic}{%
\section{Are people who believe that science is important for making
government decisions about COVID-19 more likely to disapprove of the way
their governor is handling the
pandemic?}\label{are-people-who-believe-that-science-is-important-for-making-government-decisions-about-covid-19-more-likely-to-disapprove-of-the-way-their-governor-is-handling-the-pandemic}}

\begin{Shaded}
\begin{Highlighting}[]
\KeywordTok{library}\NormalTok{(dplyr)}
\KeywordTok{library}\NormalTok{(ggplot2) }
\KeywordTok{library}\NormalTok{(tidyverse)}
\KeywordTok{library}\NormalTok{(stats)}
\KeywordTok{library}\NormalTok{(haven)}
\end{Highlighting}
\end{Shaded}

\hypertarget{importance-and-context}{%
\subsection{Importance and Context}\label{importance-and-context}}

Are people who believe that science is important for making government
decisions about COVID-19 more likely to disapprove of the way their
governor is handling the pandemic?

Science is political. Or, more accurately, science has been
\emph{politicized}.

The recent pandemic has proven beyond a doubt that topics which should
remain in the realm of logic, truth, and academic rigor can be twisted
to suit the needs of political organizations and news outlets. Where we
should find transparency we find manipulation, and where we should find
honesty we find bias.

Our current culture war has left a small but growing number of
individuals so disillusioned with the scientific process that they
reject conventional science outright. Studying the broader societal
impact of the politicization of science is our only hope of finding a
new normal that centers around truth as opposed to fear.

Our data, as described below, gives us a novel opportunity to
investigate the impact that the denial of the role of science in
decision making has on a person's view of their elected officials.
Specifically, we will explore whether or not a belief that science is
important in making decisions about COVID-19 causes people to be more
critical or their governor's handling of the pandemic. Hopefully, this
research can help reveal broader trends arising out of our healing
society.

\hypertarget{description-of-data}{%
\subsection{Description of Data}\label{description-of-data}}

We will address this question using data from the preliminary release of
the 2020 American National Election Studies (ANES). This dataset was
pooled from 8,280 pre-election interviews conducted by web, video, or
telephone. Preliminary in this context means that the data lacks some of
the variables and processing that the full report will contain.

The variables of interest in addressing this question are:

\hypertarget{v202310-in-general-how-important-should-science-be-for-making-government-decisions-about-covid-19}{%
\subsubsection{\texorpdfstring{1) \emph{V202310:} In general, how
important should science be for making government decisions about
COVID-19?}{1) V202310: In general, how important should science be for making government decisions about COVID-19?}}\label{v202310-in-general-how-important-should-science-be-for-making-government-decisions-about-covid-19}}

Respondents were given the following options for answering this
question.

\begin{verbatim}
## 
## Labels:
##  value                                                          label
##     -9                                                    -9. Refused
##     -7 -7. No post-election data, deleted due to incomplete interview
##     -6                                 -6. No post-election interview
##     -5                 -5. Interview breakoff (sufficient partial IW)
##      1                                        1. Not at all important
##      2                                          2. A little important
##      3                                        3. Moderately important
##      4                                              4. Very important
##      5                                         5. Extremely important
\end{verbatim}

In processing the data, we removed all people that did not answer this
question (responses -9, -7, -6, and -5). With these people removed we
found that respondents overwhelmingly believed that science is important
in making government decisions about the pandemic. 52.321307 percent
responded that science is extremely important, while only 1.5929204
percent responded that science is not at all important.

\hypertarget{v201145-do-you-approve-or-disapprove-of-the-way-governor-of-respondents-preloaded-state-has-handled-the-covid-19-pandemic}{%
\subsubsection{\texorpdfstring{2) \emph{V201145:} Do you approve or
disapprove of the way {[}Governor of respondent's preloaded state{]} has
handled the COVID-19
pandemic?}{2) V201145: Do you approve or disapprove of the way {[}Governor of respondent's preloaded state{]} has handled the COVID-19 pandemic?}}\label{v201145-do-you-approve-or-disapprove-of-the-way-governor-of-respondents-preloaded-state-has-handled-the-covid-19-pandemic}}

Of the 7345 respondents that answered this question 62.437032 percent
approved of their governor's handling of the pandemic and 37.562968
percent disapproved.

\hypertarget{combining-approval-of-governor-and-importance-of-science}{%
\subsubsection{Combining approval of governor and importance of
science}\label{combining-approval-of-governor-and-importance-of-science}}

\includegraphics{question_3_files/figure-latex/unnamed-chunk-3-1.pdf}

\hypertarget{flattening-the-data}{%
\subsubsection{Flattening the data}\label{flattening-the-data}}

To perform a statistical test on this data set we must divide the data
into two groups. These groups need to be

\begin{enumerate}
\def\labelenumi{\arabic{enumi})}
\tightlist
\item
  People who believe that science is important for making government
  decisions about COVID-19
\item
  People who believe that science is \textbf{\emph{NOT}} important for
  making government decisions about COVID-19
\end{enumerate}

Unfortunately, our survey participants were given five options when
asked about the importance of science. To divide our data into two
groups we chose to group together anyone that believes that science is
at least a little important in making decisions about COVID-19.

\begin{Shaded}
\begin{Highlighting}[]
\NormalTok{question3Data}\OperatorTok{$}\NormalTok{IsScienceImportant <-}\StringTok{ }\NormalTok{question3Data}\OperatorTok{$}\NormalTok{ImportanceOfScience }\OperatorTok{>}\StringTok{ }\DecValTok{1}
\end{Highlighting}
\end{Shaded}

This left us with two groups. Although the groups differ greatly in
size, we can still use a t-test to compare their sample means.

\begin{verbatim}
##                         
##                          Science isn't important Science is important
##   Disapprove of Governor                      76                 2683
##   Approve of Governor                         41                 4545
\end{verbatim}

\hypertarget{most-appropriate-test}{%
\subsection{Most appropriate test}\label{most-appropriate-test}}

Because we are comparing a binary variable (approve or disapprove)
between two samples of differing size we will use a t-test to compare
the proportion of people that approve of the governor.

This test requires 3 assumptions:

\begin{enumerate}
\def\labelenumi{\arabic{enumi})}
\item
  Metric scale: We are comparing the proportion of people that approve
  of the governor's handling of COVID-19, this is a metric variable
  between 0 and 1. Therefore, this assumption holds.
\item
  IID data: Participants in this study were selected from a random draw
  from the USPS computerized delivery sequence file (C-DSF), with all
  included residential addresses across the 50 states and Washington DC
  having equal probability of selection. A cash incentive was used to
  encourage participation. While there is undoubtedly some bias
  associated with the individuals that chose to participate in the
  survey, the selection process was fair and we've concluded that this
  portion of the sample is independent and identically distributed
  enough to warrant research on the resultant data.\n The other source
  of data for this study was everyone that participated in the 2016
  version of ANES. Emails were sent to these former participants
  inviting them to participate again. Because this portion of the sample
  is dependent on participation in the 2016 study, there is a concern
  that the data isn't independent.
\item
  No major deviations from normality: Because COVID-19 is a very recent
  development, there is not a body of research that could hint at what
  the expected distribution of this data would be. However, average
  approval of governors in general tends to be normally distributed.
  Given that our smallest group contains 117 people the central limit
  theorem should be sufficient to ensure that the sample mean is
  normally distributed.
\end{enumerate}

Because the question asks if one group is \textbf{\emph{more}} likely to
disapprove of the governor's handling of COVID-19 we will use a one
tailed test.

\hypertarget{hypothesis}{%
\subsection{Hypothesis}\label{hypothesis}}

\hypertarget{null}{%
\subsubsection{Null}\label{null}}

People who believe that science is important for making government
decisions about COVID-19 are less or equally likely to disapprove of the
way their governor is handling the pandemic.

\hypertarget{alternative}{%
\subsubsection{Alternative}\label{alternative}}

People who believe that science is important for making government
decisions about COVID-19 are more likely to disapprove of the way their
governor is handling the pandemic.

\hypertarget{test-results-and-interpretation}{%
\subsection{Test, results and
interpretation}\label{test-results-and-interpretation}}

\begin{Shaded}
\begin{Highlighting}[]
\KeywordTok{t.test}\NormalTok{(question3Data}\OperatorTok{$}\NormalTok{ApproveGovernorsHandling }\OperatorTok{~}\StringTok{ }\NormalTok{question3Data}\OperatorTok{$}\NormalTok{IsScienceImportant, }\DataTypeTok{alternative =} \StringTok{"g"}\NormalTok{)}
\end{Highlighting}
\end{Shaded}

\begin{verbatim}
## 
##  Welch Two Sample t-test
## 
## data:  question3Data$ApproveGovernorsHandling by question3Data$IsScienceImportant
## t = -6.2331, df = 119.85, p-value = 1
## alternative hypothesis: true difference in means is greater than 0
## 95 percent confidence interval:
##  -0.3524104        Inf
## sample estimates:
## mean in group FALSE  mean in group TRUE 
##           0.3504274           0.6288046
\end{verbatim}

We have failed to find any evidence to support the hypothesis that
people who believe that science is important for making government
decisions about COVID-19 more likely to disapprove of the way their
governor is handling the pandemic.

On the contrary, this data set suggests that a future experiment should
explore the hypothesis that people who believe that science is important
for making government decisions about COVID-19 are \emph{less} likely to
disapprove of the way their governor is handling the pandemic. We did
not set out to address this, and therefore we won't attempt any further
tests in this report. But we believe that testing this hypothesis could
prove more fruitful based on an exploration of our data.

\end{document}
